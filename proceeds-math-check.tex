\documentclass[10pt,letterpaper,oneside]{article}
\usepackage{amssymb,amsmath}

\begin{document}

\title{Proceeds Divided Between Two Parties\\with Recouped Costs}
\author{Jason Massey}
\date{\today}
\maketitle

% -----------------------------------------------------------------------------
\begin{abstract}
This article investigates mathematical principles when two parties agree to share proceeds in the sale of an asset,
with one party wishing to recoup invested costs.
\end{abstract}

% -----------------------------------------------------------------------------
\section{Setting}
There exist two parties, $L$ and $J$, being equal shareholders in an asset $A$ to be sold.
The proceeds $P$ from the sale are to be divided evenly after $L$ first recoups costs~$C$ (which were invested into $A$,
presumably to increase $A$'s pre-sale value).
The resultant proceeds belonging to $L$ and $J$ are denoted $P_L$ and $P_J$, respectively.

\subsection*{Constraints}
\begin{gather*}
P \ge C \ge 0               \\[10pt]
P_L \ge P_J
\end{gather*}

% -----------------------------------------------------------------------------
\section{Method to Calculate Proper Proceeds}
\begin{enumerate}
\item Begin with total proceeds $P$.
\item Immediately subtract from $P$ the costs $C$ of $L$, allocating $C$ to $P_L$.
\item The remaining proceeds, $P-C$, are to be divided equally with $\frac{P-C}{2}$ being allocated to both $P_L$ and $P_J$.
\end{enumerate}

% -----------------------------------------------------------------------------
\section{Proper Proceeds}
By following the aforementioned method, the respective \emph{proper} proceeds are:

\begin{align}
P_L &= C + \frac{P-C}{2} \\[10pt]
P_J &= \frac{P-C}{2}
\end{align}

\noindent
Note the following:

\begin{equation}
P_L = C + P_J       \label{l_e_c_plus_j}
\end{equation}


%\newpage
% -----------------------------------------------------------------------------
\section{Relationships}

\subsection{Individual proceeds sum to $P$.}
The sum of the individual proceeds for $L$ and $J$ equal total proceeds $P$:
\begin{equation}
P_L + P_J = P
\end{equation}

\noindent
proof:

\begin{align*}
P_L + P_J &=  \\[10pt]
\left(C + \frac{P-C}{2}\right) + \left(\frac{P-C}{2}\right) &= \\[10pt]
C + \frac{P-C}{2} + \frac{P-C}{2} &= \\[10pt]
C + 2\left( \frac{P-C}{2} \right)  &= \\[10pt]
C + \left( P-C \right)  &= \\[10pt]
C + P-C &= \\[10pt]
P & \quad\blacksquare
\end{align*}

\newpage
\subsection{Subtracting $P_J$ from $P_L$ leaves $C$.}
To confirm that $L$ recoups costs $C$ entirely, it must be shown that:

\begin{equation}
P_L - P_J = C       \label{l_m_j_e_c}\\[10pt]
\end{equation}

\noindent
Although (\ref{l_m_j_e_c}) follows naturally from (\ref{l_e_c_plus_j}), it is nevertheless proven below.

\begin{align*}
P_L - P_J &= \\[10pt]
\left(C + \frac{P-C}{2}\right) - \left(\frac{P-C}{2}\right) &= \\[10pt]
C + \frac{P-C}{2} - \frac{P-C}{2} &= \\[10pt]
C + \left(\frac{P-C}{2}\right)(1-1) &= \\[10pt]
C + \left(\frac{P-C}{2}\right)(0) &= \\[10pt]
C & \quad\blacksquare
\end{align*}


\newpage
\subsection{When the costs to be recouped $C$ are zero, $P_L$ and $P_J$ are both equal to split proceeds.}
Given (\ref{l_m_j_e_c}), it follows when $C=0$ that:

\begin{align*}
P_L - P_J &= 0                      \\[10pt]
P_L &= P_J                          \\[10pt]
C + \frac{P-C}{2} &= \frac{P-C}{2}  \\[10pt]
0 + \frac{P-0}{2} &= \frac{P-0}{2}  \\[10pt]
\frac{P}{2} &= \frac{P}{2}          \quad\blacksquare   \\[10pt]
\end{align*}


\newpage
\subsection{If $L$ and $J$ are forced to equally split $P$, then $J$ shall owe to $L$ exactly $\frac{C}{2}$.}

Closing coordinators at title companies might not be aware of $L$ and $J$'s agreement to have $L$ recoup
costs $C$ before evenly sharing the remaining proceeds $P-C$.
In this case, $L$ and $J$ would be forced to split $P$ in half,
receiving \emph{improper} proceeds $P_i$ of:

\begin{equation}
P_i = \frac{P}{2}       \\[10pt]
\end{equation}

\noindent
with the notable property given $C>0$:

\begin{equation}
P_i < P_L       \\[10pt]
\end{equation}

\noindent
In this situation, in order to make $L$'s proceeds \emph{proper,} $J$ would need to pay a reconciliation $R$ to $L$ computed as:

\begin{equation}
R = P_L - P_i       \\[10pt]
\end{equation}

\noindent
namely,

\begin{align*}
R &= P_L - P_i                                                      \\[10pt]
&= \left( C + \frac{P-C}{2} \right) - \left(\frac{P}{2}\right)      \\[10pt]
&= C + \frac{P-C}{2} - \frac{P}{2}                                  \\[10pt]
&= C + \frac{P-C-P}{2}                                              \\[10pt]
&= C + \frac{-C}{2}                                                 \\[10pt]
&= \frac{2C}{2} + \frac{-C}{2}                                      \\[10pt]
&= \frac{2C-C}{2}                                                   \\[10pt]
&= \frac{C}{2}      \quad\blacksquare                               \\[10pt]
\end{align*}


\end{document}

% -----------------------------------------------------------------------------
% vim: tw=0 nowrap et
